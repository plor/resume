% LaTeX resume using res.cls
\documentclass[line,margin]{res} 

\begin{document}

\name{Jordan I. Walker}
% \address used twice to have two lines of address
\address{104 14th Street, Prairie du Sac, WI 53578}
\address{email: jordan@jordanwalker.us | phone: 608.370.1908}
 
\begin{resume}
 
\section{BACKGROUND}
	\textbf{Computer Scientist} with over \textbf{eight years} experience in \textbf{Java} back-end programming and various languages in \textbf{scripting} and \textbf{web development}.
	Specializes in the \textbf{agile} process of \textbf{planning}, \textbf{implementation}, and \textbf{delivery} of high quality software suited to the needs of the customer.
	Particularly familiar with the challenges of working with \textbf{data services} in \textbf{aquiring}, \textbf{processing}, and \textbf{distributing} data efficiently and effectively.
	Brings together the traditional role of \textbf{software engineer} and nascent role of \textbf{data scientist} by bringing the solid \textbf{system-level} understanding of computer science to the practical \textbf{data-driven} analysis for problem solving in scientific fields that need both.


\section{TECHNICAL SKILLS}
	Particularly skilled at learning additional languages and frameworks, below are skills I have to date:
\begin{itemize}
	\item Proficient in \textbf{web application} programming across several languages, including \textbf{Java}, \textbf{Perl}, and \textbf{R/shiny}.
	\item Proficient with \textbf{Agile} methodologies for product delivery, including agile planning, iterative development, and continuous delivery.
	\item Proficient in \textbf{R} programming and the \textbf{Rstudio} environment for both package and script development.
	\item Proficient with R packages developed by USGS and EPA for \textbf{data discovery} and \textbf{data analysis} in order to \textbf{produce insights} based on these data.
	\item Experienced in visualization techniques in R, such as \textbf{base plotting} and \textbf{ggplot2} to aid in analysis and presentation of results.
	\item Proficient in version control, particularly \textbf{git} and \textbf{github}, for software as well as scientific workflows.
	\item Familiar with low-level languages like \textbf{C}, \textbf{C++}, and \textbf{FORTRAN}, particularly when it is useful to employ them for efficiency and performance.
	\item Experienced with relational database management using \textbf{SQL} and related technologies, including \textbf{hibernate}, \textbf{myBatis}, \textbf{liquibase}.
	\item Familiar with \textbf{python} as a scripting, analysis, and application language.
	\item Proficient in \textbf{Linux} server and desktop management, including \textbf{bash} scripting for everyday uses.
	\item Experienced with the client-side technologies of the web, \textbf{HTML5}, \textbf{javascript}, and \textbf{CSS}.  This includes many common libraries and frameworks such as \textbf{jQuery}, \textbf{LESS}, \textbf{angularJS}, and \textbf{d3}.
	\item Familiar with \textbf{\LaTeX} for typesetting and building attractive documents.
	\item Proficient working with scientific data file types, services, and encodings, including \textbf{NetCDF}, \textbf{OPeNDAP}, \textbf{RDB} and \textbf{WaterML2}.
	\item Proficient in automated testing tools in several languages, particularly \textbf{JUnit} in Java and \textbf{testthat} in R.
	\item Experience with R language \textbf{training}, having assisted in development of a curriculum and instruction for several \textbf{scientific computing} courses.
	\item Experienced in formal dependency management solutions in several languages, starting with \textbf{maven} within the JVM ecosystem and extending to the different options built around \textbf{CRAN} within R.
	\item Proficient in project management following Agile practices, examples include \textbf{planning poker} for release planning, \textbf{SCRUM} and \textbf{kanban} for iteration planning and execution, and \textbf{retrospectives} for continuous process improvement.
	\item Experienced with \textbf{devops} tools and principles as a way of extending development further towards the infrastructure and production environment.
\end{itemize}
 
\section{PROFESSIONAL EXPERIENCE}
		{\sl Independent Computer Scientist} \hfill Oct 2017 - Present \\
		\begin{itemize} \itemsep -2pt % reduce space between items
			\item Provided \textbf{web development} services for clients.
			\item Learned \textbf{solidity} programming language and developed Ethereum blockchain smart contracts.
			\item Focused on research and development into emerging technological domains and bootstrapping to engage with them.
			\item Working on an in-progress \textbf{Agile} facilitation program to push the boundaries of this practice.
		\end{itemize}

		{\sl Computer Scientist} \hfill June 2010 - Sep 2017 \\
		U.S. Geological Survey \hfill Middleton, WI

		Worked on a software engineering team developing services for USGS Water data, followed by a transition to a newly established Data Science team.  Played a lead role in many projects as well as a supporting role in many others. 

	     \begin{itemize} \itemsep -2pt % reduce space between items
		\item Member of Water Mission Area \textbf{data science team} focused on developing \textbf{tools}, \textbf{training} in scientific computing, performing \textbf{research}, and communicating science with compelling \textbf{visualizations}.
		\item Developed Java-based \textbf{Geo Data Portal} for accessing USGS \textbf{downscaled climate data}.
		\item Worked closely with domain experts to assist in several areas of Water science (\textbf{surface water}, \textbf{groundwater}, \textbf{water quality}, and \textbf{water use}).
		\item Architect and supporting developer of \textbf{geoknife} R package for accessing the Geo Data Portal.
		\item Project lead on USGS \textbf{Visualization Laboratory} and the accompany \textbf{vizlab} R package.
		\item Worked on several projects using vizlab, dataRetrieval and geoknife packages to produce \textbf{visualizations} for the general public.
		\item Researched and implemented applications using domain specific software (THREDDS, geoserver, 52 North WPS).
		\item Designed services and clients supporting \textbf{open standard} data exchange (WFS, WMS, CSW, SOS, OPeNDAP).
		\item Co-created web portal for \textbf{Coastal Change Hazards Portal} assessing risk of the nation's coast to different hazards such as storms and sea-level rise.
		\item Created web user interfaces using JavaScript and well known libraries and frameworks (jQuery, angular, openlayers).
		\item Worked in a team environment using \textbf{Agile practices} for planning, implementation, and delivery.
	     \end{itemize}

		{\sl Student Programmer} \hfill Feb 2007 - May 2010 \\
		University of Wisconsin Space Science Data Center \hfill Madison, WI
                
		Student programmer to the data center operations staff.  Tasked with a variety of programming tasks to make operations more efficient.  Worked closely with staff to define requirements of scripts and dashboards that were developed.

	\begin{itemize} \itemsep -2pt % reduce space between items
                 	\item Supported operation of \textbf{600 Terabyte} datacenter.
		\item Developed web applications \textbf{dashboards} for quality control of incoming satellite data.
		\item Created scripts for application and system \textbf{monitoring} for operational systems. 
		\item Wrote and updated programs working with several \textbf{mySQL} databases with metadata about datacenter contents.
                 	\item Developed Java code for NASA Atmosphere PEATE project including \textbf{data ingestion}.
		\item Programming languages used were \textbf{Perl}, \textbf{python}, \textbf{Java}, and \textbf{PHP} along with some web programming in HTML, JavaScript and CSS.
                \end{itemize}

\section{SELECTED PUBLICATIONS}
	\begin{itemize} \itemsep -2pt
		\item {\sl Smartphone-Based Distributed Data Collection Enables Rapid Assessment of Shorebird Habitat Suitability.} Thieler, E. Robert; Zeigler, Sara; Winslow, Luke; Hines, Megan; Read, Jordan; Walker, Jordan. PLoS ONE, 2016.
		\item {\sl geoknife: Reproducible web-processing of large gridded datasets.} Read, Jordan; Walker, Jordan; Appling, Alison; Blodgett, David; Read, Emily; Winslow, Luke. Ecography, 2015.
		\item {\sl Description of the US Geological Survey Geo Data Portal Data Integration Framework.}  Blodgett, David; Booth, Nathaniel; Kunicki, Tom; Walker, Jordan; Lucido, Jessica. IEEE, 2012.
		\item {\sl A system for audio signalling based NAT traversal.} Patro, Ashish; Ma, Yadi; Panahi, Fatemeh; Walker, Jordan; Banerjee, Suman. COMSNETS IEEE, 2011.
     		\item {\sl Continuous Monitoring of Wide-area Wireless Networks: Data Collection and Visualization.} Ormont, Justin; Walker, Jordan; Banerjee, Suman. Sigmetrics Performance Evaluation Review, 2008.
		\item {\sl A City-wide Vehicular Testbed for Wide-area Wireless Experimentation.} Ormont, Justin; Walker, Jordan; Banerjee, Suman; Sridharan, Ashwin; Seshadri, Mukund; Machiraju, Sridhar. WiNTECH, 2008.
	\end{itemize}

\section{EDUCATION} {\sl MS,} Computer Science \\
        	     	University of Wisconsin-Madison -- Madison, WI \\ 
                	May 2010, GPA 3.75/4.0

		{\sl BS,} Computer Science \\
		University of Wisconsin-Madison -- Madison, WI \\
		Graduated with Honors, May 2008, GPA 3.623/4.0

\end{resume}
\end{document}
